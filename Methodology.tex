\section{Methodology}
Sense and Avoid Identification System\\ \\
\indent The SAA Identification System (See Appendix A) will be a system that incorporates stereoscopic and aerial morphological identification to produce a hybrid system yielding greater results in efficiency and accuracy for UAS flight paths and object classification. The Boolean indication of an object from the stereoscopic subsystem activates the morphological subsystem (See Appendix A: Morphological Identification). The hybrid system will use the counterbalancing influences from the subsystems to further verify the SAA identification system’s final decision.\\ \\                 
Morphological Identification\\ \\
\indent Morphology identification use inverse background subtraction. The goal is to find the foreground object and its steady state in contrast to the transient portions of the centered aerial object. The foreground object full unprocessed primary profile is averaged with the secondary frame’s object profile and given a ratio based against the objects processed steady state profile. This ratio equals the area of the object’s steady state over the averaged full profile of the object. The bias threshold was found based upon test data. The concept is the morphology of an aerial biological fluctuates far more rapidly than typical manmade aerial vehicles during active thrust and lift.\\
\indent Morphology of a manmade aircraft contrast to an aerial biological, are much steadier. The more notable parts of a fixed wing aircraft are the wings and the fuselage which remain stationary in relation to the entire profile. The noticeable transient portion of the fixed wing aircraft is its propeller for its constant motion. The aliased propeller captured in the video stream is smaller compared to the main components of its profile. A rotorcraft body profile is identified through its main rotor and fuselage. The aliased main rotor captured in the video stream is small in comparison with its fuselage. The aerial biological is set apart by its organic movements and motions. Aerial biologicals are commonly identified through the motion of their wings. The motions and movements of the aerial biological contrasted with traditional manmade vehicles are appreciably evident. Manmade aerial objects maintain a higher steady state profile area despite various methods of thrust and lift compared to aerial biologicals.\\
\indent The experimental platform uses a computer workstation and open source computer vision software to test the hypotheses in this study. The computer workstation is a 64-bit Ubuntu 16.04 linux based computer work station equipped with Python 2.7 and Open Source Computer Vision(OPENCV) 3.0 with python equipped libraries. The completed SAA identification system (outside of this study) is designed to run on an ARM core unix-based Raspberry Pi 3 Model B microcomputer with integrated Raspberry Pi NOIR Rev.1.3 cameras. In order to stay within the parameters of the physical hardware of the completed system a unix-based computer was utilized for the duration of the study. The open source computer vision software used in the study also has been experimentally tested and found to work with the linux-based microcomputer. The experimental platform uses systems and programming that have the ability to be directly transferred to the microcomputer upon completion of the system.\\           
\indent The replication of the study involves a simulated object tracking, noise reduction, and binary thresholding. Each sample is a grayscale image of a single aerial object with a light blue region as background centered in the video stream as if object tracking is being applied. The maintained centering of the object in the video stream is mandatory for overlay in the process of Boolean image multiplication. Boolean image multiplication uses time varying frames from the video stream to yield the steady state profile.\\ 
\indent Noise reduction is accomplished through image smoothing, which uses a 5x5 Gaussian normal distribution to eliminate the noise from the background. Noise in the video stream includes darker areas and undesirable visibility of clouds, particles in the sky, and very fine details that would show unwanted variations in the illumination of the foreground aerial object.\\
\indent Binary thresholding is a technique that applies a replacement of pixel value (intensity) to 255 (white) or 0 (black) depending on the state of the original pixel value as above or below a pixel threshold. This process is used to further illuminate the aerial object by providing higher contrast to the video stream frames. The binary threshold is used to separate the foreground object from the background sky region.\\
\indent The featured process in this study is Boolean image multiplication process, which calculates the steady state ratio. The steady state ratio is a ratio calculated from the steady state profile of the aerial object over the time duration in which the frames were acquired (seen in Equation 1) 

\begin{equation}
SSP = r_g^1(t) \times \frac{r_g^2(t')}{r_{g max}} 
\end{equation}

and the average area of the full profile of the aerial object from the interval varying frames used in the process (seen in Equation 2). 

\begin{equation}
SSR = \frac{SSP_{Area}}{\overline{FP_{Area}}} 
\end{equation}

\indent The Area of the objects steady state is acquired and a ratio is created between the area of the steady state to the average area of the entire signal at the two different time periods. The ratio calculated in the process is used to classify the object through the amount of change in its structure while maintaining active thrust and flight. The ratio calculated from the object’s morphology is then correlated to either aerial biological object or aerial manmade object separated by a bias threshold.\\
Hypothesis \\ \\
Null Hypothesis ($H_0$): There will be no change between the steady state ratios of the manmade versus the Steady State Ratios of the aerial biological objects. \\ \\
Alternative Hypothesis ($H_A$): There will be a directional change between the steady state ratios of the manmade versus the Steady State Ratios of the aerial biological objects. \\

The Hypothesis used to define the study is searching for a change in the groups of video sample frames that shows a significant difference in the manmade aerial objects from the aerial biological objects. 

\begin{center}
$
H_0: SSR_M \leq SSR_B $\\$
H_A: SSR_M > SSR_B
$
\end{center}

\indent The quantitative research model was used to research the differentiation of the aerial objects. The steady state ratios were directly relatable to the question of distinguishing objects through their morphology by the mean values of each dataset(Aerial Biological and Aerial Manmade). The statistical analysis chosen to decipher whether a change could be measured is the Heteroscedatics One-tailed Paired T-test. The One-tailed paired t-test is a statistical tool used to analyze the mean differences between datasets to determine whether one dataset is significantly greater than another. \\ \\
\indent Using Boolean image multiplication an outcome can only come from the significant difference in the steady state ratios created from the video sample subjects. The assumption that a manmade aerial object has a higher steady state ratio than an aerial biological object is used in the hypothesis to gain further depth into the understanding of the resulting probability and application.\\
\indent The data sets used in the study were acquired from Instagram footage. There were 399 steady state ratios collected from the process of Boolean image multiplication. 21 different variations of frames were used in each video sample. 19 video samples of aerial objects at 30 fps (.033s between frames) were chosen for the study. 252 steady state ratios were collected from all variations of corresponding video samples pertaining to aerial biologicals. 147 steady state ratios were collected in all frame variations from corresponding video samples that contain manmade aerial objects. Due to the unequal variability of the datasets, a Heteroscedastic Paired T-test is required.\\     
\indent There are limitations of this model system. This study is the preliminary work toward improvements to the SAA and DSA system with the respect to threat avoidance efficiency. The current limitations of the system are the ability to process and identify only one object per frame, against a background of a light blue region unobstructed by landscape. The ability to detect based on the size of the bird due to the surface area of the wing and the frequency needed to keep the biological aloft. This method is limited to biologicals actively providing thrust; soaring or gliding observed through the system may not accurately be classified correctly. The system is limited by the intensity of the hue of the aerial objects; any aerial objects emitting (color or reflectivity) intensities analogous to the sky region may be filtered out as noise.\\ 
\indent This study assumes that the video stream sample aerial biological object represents all aerial biological objects. The sample objects used in the study as manmade vehicles span only fixed wing aircraft and single main/tail rotor rotorcraft. The intensity of the hue of both manmade and biological objects are assumed to be a lower magnitude than the hue intensity of the sky region backdrop. The study only looks at aerial objects located in the main forward facing field of view to be perceived as either a “chase” object or an object approaching “head-on”. The object’s flight path is assumed to be steady straight and level flight.\\
\newpage