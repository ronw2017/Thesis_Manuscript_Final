\section{Background}
\indent The popularity of unmanned aerial systems among the population of the United States is rising significantly every year \cite{[5],[7]}. “UAS hold enormous promise for our economy and for the aviation industry. But for the industry to develop to its full potential, we have to ensure that it develops safely” \cite{[5]}. The technology is becoming easier to acquire and use. Due to the high volume of unmanned aerial system sales and the high number of untrained users, UAS began to unintentionally intrude the NAS potentially creating hazardous environments for authorized traffic. $"$Air traffic control (ATC) received increasing reports of unauthorized and unsafe use of small UAS. Pilot reports of UAS sightings in 2015 are double the rate of 2014. Pilots have reported seeing drones at altitudes up to 10,000 feet, or as close as half-a-mile from the approach end of a runway$"$ \cite{[5]}. The safety of the National Airspace System prompted the Federal Aviation Administration to create restrictions \cite{[1],[2],[3],[5],[6]} and a registration system \cite{[1],[2],[5],[6]} and to prepare a complete set of regulations \cite{[4],[7],[8],[9]} for these unmanned aerial systems.\\ 
\indent The personal and commercial potential of this disruptive technology was recognized by corporations, civilians, and the government \cite{[1],[5]}. The price of the technology in comparison to equivalent manned equipment is significant and the ever-widening variety of the functions the technology can be applied to could not be ignored \cite{[1],[5]}. Corporations and government are pressing for the expedient establishment of the regulations, to be granted the authorization to use these unmanned aerial systems as powerful assets. The establishment of the regulations was the initial step towards the safe use of UAS in the NAS. Difficulty arose with the resulting hazard branching from the inability for an unmanned system to avoid collisions with other aircraft, which resulted in the call for a “Sense and Avoid capability” \cite{[2],[7],[8],[9]}. “Sense and avoid capability means the capability of an unmanned aircraft to remain a safe distance from and to avoid collision with other airborne aircraft” \cite{[2]}.\\ 
\indent Sense and avoid capability exists and is currently in use aboard airlines and high-end aircraft. The current sense and avoid capable technologies in use are cooperative avoidance systems (TCAS and ADS-B); cooperative avoidance means that the avoidance systems work together by broadcasting their location and trajectory to other cooperative avoidance systems in the area calculating and alerting the crew members on a possible collision path \cite{[8],[9]}. The ADS-B and TCAS sense and avoid technologies have a native weakness, the nature of cooperative systems \cite{[8],[15],[16]}. Cooperative avoidance systems can only sense and avoid other cooperative avoidance systems. These cooperative avoidance systems are not required by law to be used by the entire population of the NAS, which makes unequipped aircraft invisible to the current systems and requires periodic human monitoring. Secondary complications with the system that limit its full integration into the population of the NAS include its cost, weight, size, and power consumption \cite{[8],[14],[15],[16]}. This prompted research into a sense and avoid capable system that could be used with non-cooperative systems.\\ 
\indent The evolution of the non-cooperative sense and avoid started with a Boolean search and avoidance response. The system detected an object in the field of view of the sensor and immediately avoided the object \cite{[9],[10],[11]}. This initial step proved that the system was responsive. Improvements were further studied and implemented to sense the depth and size of the object detected. This improvement used the comparison of the size of the aerial object to the size of the host unmanned aerial system to send an indication to the avoidance system that the object should be avoided \cite{[9],[10]}. This improvement modestly explored the threat level of aerial objects based on size. The threat based solely on size is unsatisfactory, as UAS, birds, and aircraft come in many different sizes and forms \cite{[17]}.\\ 
\indent The next iteration of improvements incorporates the trajectory of the detected aerial object. The size (referenced to host UAS), depth, and trajectory of the object are correlated with the velocity of the host unmanned aerial systems and the sense system calculates the potential for a collision given the two objects (host UAS and foreign object) present course \cite{[12],[14],[15] ,[16]}. This iteration of improvement explores the size (relative to self) and trajectory as a level of threat. This system has the potential to avoid a collision of an object with steady heading and attitude, but the potential for an object to rapidly change the direction in all the sky is high and the process is computation intensive. The sense and avoid system’s development then started pivoting towards hybrid systems that analyze a fusion of sensors to improve the accuracy of the system in different environmental situations by comparing resulting data about the aerial object detected \cite{[13],[14]}. The fusion studies focus on the abilities of multiple sensors for their variety of strengths to offset the weaknesses of the other sensors. This sensing fusion was used to improve abilities in low light scenario and visibly inconvenient weather conditions. The advantage of using a system of checks and balances is the increased accuracy.\\ 
\indent The exploration of the research of sense and avoid systems has brought several improvements; the under researched areas are the over sensitivity in the sense systems to aerial objects, the inability to handle rapidly maneuvering aerial objects, and the foresight to understand the complications that a highly-populated airspace introduces for this type of system.                                                         
