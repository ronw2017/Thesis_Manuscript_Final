\section{Abstract}
The future of aviation is moving toward complete autonomy; Corporations are pursuing permissions to operate unmanned aerial systems (UAS) in the National Airspace System (NAS) under semi-autonomous to full autonomous control, and an increase in the number of UAS being used by civilians in the private sector necessarily means that the skies are soon to be densely populated with these systems. One way to accommodate the increasingly high population of UAS in the NAS is to improve the efficiency of the flight paths of these systems. These flight paths involve either altering the distance of separation (FAA regulated) or improving the detection of the system to avoid and take positive evasive action from potentially dangerous objects while detecting, but not necessarily navigating away from the benign objects. The goal is to increase efficiency of the sense and avoid detection system while maintaining safety by positively determining the threat level of an aerial object. “Dangerous” aerial threats include manned aerial vehicles, while “benign” aerial threats include aerial biologicals. These are classified as benign because of their capacity for agile movements and curious/protective nature over their territory. Using Boolean Image multiplication to distinguish various aerial objects, we were able to obtain a confidence level of 99.847\% that there is a noticeable difference between the aerial objects using steady state analysis. These differentiations are stepping stones toward an enhanced Detect, Sense, and Avoid system using morphological analysis to increase efficiency.   
