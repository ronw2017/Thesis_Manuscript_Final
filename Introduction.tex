\section{Introduction}
\indent The importance of detect, sense, and avoid subsystems in unmanned aerial system(UAS) is the safe and efficient operation of UAS in the National Airspace System(NAS) is becoming increasingly critical. This study focuses on the improvement in detection ability of non-cooperative sense and avoid and detect, sense, and avoid systems. The non-cooperative sense and avoid system is a system that can be entirely contained onboard a UAS giving it the capability to remain a safe distance from other airborne aircraft and avoid collision. This improvement of the ability to classify aerial objects helps the integration of UAS into the NAS through flight path efficiency, which can potentially increase the capacity for UAS in the NAS. The increase in flight path efficiency is attained through the classification of threat levels, which may allow for a much more passive flight path adjustments to deal with benign threats. Aerial biological objects are considered benign aerial threats due to their natural maneuverability and their protective curiosity of foreign aerial objects in their territory. The study examines the differentiation between manmade biological aerial objects. The classification of aerial objects is achieved through time varying morphological analysis using Boolean image multiplication. Throughout the discussion and suggestions of the topic of non-cooperative UAS sense and avoid systems, many incremental improvements have been introduced to make the integration of UAS into the NAS possible. Initial systems gave Boolean responses due to the presence of any aerial objects in the host UAS field of view [9] [10] to the current stages of the systems tracking objects and calculating collisions [14] [15]. Research into a system that will comply with the high density of UAS using the ability to calculate and omit non-threatening aerial objects is the next logical iteration.\\
\indent This study will present data showing the ability of Boolean image multiplication to classify an aerial object as a manmade or biological object. The contribution to sense and avoid subsystem is improved efficiency through classification of dangerous versus benign aerial threats.